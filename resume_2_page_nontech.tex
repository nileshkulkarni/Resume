\usepackage{hyperref}
\usepackage[hmargin=2cm,vmargin=2cm]{geometry}
\documentclass[margin,11pt]{resume}

\begin{document}
\begin{resume}

\vspace{45mm}

\section{\mysidestyle Awards \&\\ Achievements}

\begin{list2}
\item \textbf {All India Rank 77} at Joint Entrance Exam (JEE) 2011, conducted by Indian Institute of Technology, among 5.5 lakhs candidates. 
\item \textbf {All India Rank 407} in ISAT 2011 conducted by Indian Institute of Space of Technology 
\item Selected for the Indian National Olympiads : \textbf{INPHO}(Physics), \textbf{INCHO}(Chemistry) conducted by HBCSE, National \textbf{Top 1\%} in each of them
\item Accomplished \textbf {Ganit Pradnya State Level Mathematics Examination} conducted by BrihanMumbai Ganit Adhyapak Mandal
\item Awarded \textbf{Institute Technical Special Mention}, 2013 for excellence in technical activities
\item Awarded Special Mention at the Performing Arts Festival(PAF), 2012
				
\vspace{-3mm}
\end{list2}\vspace{0.25mm}
\section{\mysidestyle Positions of\\ Responsibility}

\begin{list2}
\item \textbf{Technical Mentor} \hfill\textbf{2013}\\
				Mentored teams for technical summer projects under Student Technical Activity Body(STAB)
\item \textbf{Coordinator, Electronics Club,IIT Bombay} \hfill \textbf{2012}\\
				Monitoring and mentoring technical activities all over the Institute. The body conducts lectures and sessions on Technical topics
\end{list2}
\vspace{-3mm}

\section{\mysidestyle Extra\\Curricular\\Actvities}
\begin{list2}
\item Participated in  Yahoo HackU  2013, created a web-appilication which could provide car-pooling services
\item Created appilication to  auto compelete and correct song metadata, at Windows 8 Appfest 2012. Runnerup at Institute Hackathon conducted by WnCC
\item Completed the enduro3 night trek, 16kms long conducted by NEF
\item Among Top 30 players in Badminton of my Batch (selected for National Sports Organisation (NSO)); Represented Hostel in Badminton, Football, Swimming
\end{list2}

\vspace{-3mm}

\section{\mysidestyle Research Intern}
				\textbf{Visiting Scientist at  Technische University of Braunschweig, Germany\\} \emph{Online Triangulation and Navigation Using a Swarm of Simple Robots}\\
				\emph{Guide: Prof. Dr. Sandor P. Fekete \& Prof. Dr. Alexander Kroller} \hfill {\textbf{2013}} \vspace{-4mm}\\
				\begin{list2}	
				\item Researched on Maximum area triangulation problem involving \emph{K} robots, and Minimum Robot Area Triangulation Problem
				\item Contributed to practical aspects of real life problems involving minimizing overall error in navigation and localization, given minimum sensing capabilities such as Infrared Sensors
				\item Minimized the error by a significant amount making it possible for the robots to locate and align with their neighbours
				\end{list2}
\vspace{-3mm}
				\section{\mysidestyle Projects}
\textbf{AUVSI ROBOSUB, International Underwater Robotics Competition, San Deigo, California } \hfill www.auv-iitb.org
\vspace{1mm}\\ 
					\textsl{Guide : Prof.Dr Hemandra Arya and Prof Leena Vachhani}\hfill \textbf{July 2012-ongoing}\\Designing and developing a an unmanned autonomous underwater vehicle (AUV) that localizes itself and performs 	realistic missions based on feedback from visual, inertial, acoustic and depth sensors 	using thrusters\/ propeller
				Working under the Software Subdivision of 6 DOF Autonomous Underwater Vehicle 
				\begin{list2} 
				\item Implemented Navigation using Markov Localization with Visual \& acoustic feedback
				\item Using \textbf{Robot Operating System(ROS)} as platform for the software stack
				\item Designed and Implemented the a debug platform to monitor the current status of the vehicle while providing a on-board programming ability
				\item Finished \textbf{10\textsuperscript{th}} among 33 teams, at Robosub 2013\vspace{-2.75mm} 
				\end{list2}
\textbf{Ping Pong on FPGA}\vspace{1mm} \hfill\\
\textsl{Guide : Prof. Ashwin Gumaste} \hfill \textbf{Spring 2013} \vspace{-4mm}\\ 
		 \begin{list2}
		 \item Designed and implemented a \emph{double-player} version of Ping Pong on a Field Programmable Gate Array, and also interfaced it with a \textbf{HDMI} display.				\item  Used manual inputs from the user to provide paddle movements
				\end{list2}
\vspace{-2.75mm} 

\textbf{N-Body Simulation}\vspace{1mm} \hfill\\ 
\textsl{Guide: Prof. Varsha Apte}\hfill \textbf{Autumn 2013}\vspace{-5mm}\\
				\begin{list2}	
				\item Designed an simulation which showed the interaction between different particles under the effect of intermolecular forces like gravity, electrostatic and nuclear
				\item The simulation involved using the famous \emph{Barnes-Hut Algorithm} to optimize on computation
				\end{list2}\vspace{-2.75mm} 

\textbf{GIS Contour Plotting}\vspace{1mm} \hfill\\ 
\textsl{Guide : Prof. Milind Sohoni and Prof. Adsul Bharat}	\hfill \textbf{May 2012}\vspace{-4mm}\\
				\begin{list2}
				\item Plotted elevation and contours by obtaining sample data from \textbf{Google's API} using various kriging and interpolation techniques. 
				\item Anlyzed them for their complexity and implemented them efficiently in python
				\item Generated KML files and overlaid them on Google Earth for 2D and 3D views
				\end{list2}				\vspace{-2.75mm} 


\textbf{Statistical Data Analysis}\vspace{1mm} \hfill\\ 
\textsl{Guide : Prof. Milind Sohoni}	\hfill \textbf{Spring 2012} \vspace{-5mm}\\
				\begin{list2}
				\item Statistically interpreted Census (2011) data for two suburbs of Mumbai
				\item Plotted various graphs in Scilab to understand the data pictorially
				\end{list2}
				\vspace{-2.75mm} 

    \textbf{Chinese Checkers}\vspace{1mm} \hfill\\
		\textsl{Guide : Prof. Amitabha Sanyal} \hfill \textbf{Spring 2012} \vspace{-5mm}\\
				\begin{list2}
				\item
				Developed a \emph{multi-player} game of Chinese Checkers in Plt-Scheme
				\item Implemented the popular min-max and alph-beta pruning algorithm besides other algorithms designed by us for the computer player
				\end{list2} \vspace{-2.75mm} 
\textbf{All Terrain Vehicle}\vspace{1mm} \hfill
\textbf{Summer 2011} \vspace{-5mm} \\
		 \begin{list2}
    \item
		 Built under the \emph{Electronics Club Summer Project} at IIT Bombay, an All terrain vehicle, capable of changing its shape  and propelling through rough terrain
		 \item The robot can be wirelessly controlled using a \textbf{DTMF(Dual Tone Multi-Frequency Technology)}. Awarded \textbf{Best Institute Technical Project}
\end{list2} \vspace{-1.75mm}  

%\vspace*{5mm}
\section{\mysidestyle Key\\ Courses\\ Undertaken}

\textbf{Computer Science} :
Operating Systems**, Compilers**, Artificial Intelligence**, 
Computer Architecture*, Databases*
Linear Optimization*, 
Design and Analysis of Algorithms, 
Discrete Structures, 
Automata Theory and Logic\\       
\textbf{Mathematics}: Linear Algebra, Numerical Analysis, Differential Equations\\
\textbf{Electrical}: Analog Electronics, Signals \& Systems, Controls \& Communication, Introduction to Electric and Electronic Circuits\\
\textbf{Others}: Modern Physics, Data Analysis and Interpretation, Economics
\\Note: The * marked courses will be completed by November 2013 ** marked courses will be completed by April 2014.

\vspace{-1mm}
\section{\mysidestyle Skills}
\begin{list2}
\item Programming Languages: C++/C, Prolog, Python, Plt Scheme, Java, Javascript,Php
\item Microcontroller: Intel 8085, AVR, PIC, Raspberry Pi, Arm-Cortex M4 and  FPGA's
\item Platforms: Linux, Robot Operating Systems(ROS)
\end{list2}	
\section{\mysidestyle Interests} 
\begin{list2}
\item Particle Physics and Quantum Theory, Algorithm Design, Robotics
\item Hackthons and Puzzles
\item Trekking 				
\end{list2}

\end{resume}
\end{document}
