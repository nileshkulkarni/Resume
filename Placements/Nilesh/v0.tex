\documentclass{article}
\usepackage[margin = 0.75in]{geometry}
\usepackage{graphicx}
\usepackage{amsmath}
\usepackage{array}
\usepackage{enumitem}
\usepackage{wrapfig}
\usepackage{titlesec}
\newcommand{\xfilll}[2][1ex]{
\dimen0=#2\advance\dimen0 by #1
\leaders\hrule height \dimen0 depth -#1\hfill}
\titleformat{\section}{\Large\scshape\raggedright}{}{0em}{}
\renewcommand\labelitemi{\raisebox{0.4ex}{\tiny$\bullet$}}
\renewcommand{\labelitemii}{$\cdot$}
\begin{document}
\vspace*{2.0cm}
\section*{Interests\xfilll[0pt]{0.5pt}}
\vspace{-7pt}
    Machine Learning, Robotics, Artificial Intelligence
\vspace{-13pt}
\section*{Scholastic Achievements\xfilll[0pt]{0.5pt}}
\vspace{-7pt}
	\begin{itemize}[itemsep = -0.75 mm, leftmargin=*]
		\item Secured All India Rank 77 in IIT-JEE 2011, among 5 lakh candidates
        \item Awarded Institute Technical Color(5 of 7000) and Institute Technical Special Mention for outstanding contribution to Robotics and technical activities on campus
		\item Certified as among top 1\% (300 students) in India, to appear for the Indian National Chemistry Olympiad (INChO), 2011 and the Indian National Physics Olympiad (INPhO), 2011
		\item Pursuing Honors in Computer Science with Minors in Electrical Engineering
	\end{itemize}
\vspace{-17pt}
\section*{Internships\xfilll[0pt]{0.5pt}}
\vspace{-7pt}
\textbf{Samsung Electronics, Suwon} \hfill{\sl \small Summer 2014}\\
Machine Learning on Big Data Sets\hfill{\sl \small Guide: Choonoh Lee, Senior Engineer}\\
\vspace{-17pt}
\begin{itemize}[itemsep = -0.75 mm, leftmargin=*]
    \item Improved  Mahout's State of Art Implementation of K-Means Clustering , and produced results with better execution time: 6x in Pre-processing and 1.8$x$-2.2$x$ per Iteration
    \item Implemented a distributed version of Decision Forest on Hadoop Version 2,with high accuracy for classification 
    \item Published a white paper with results relating to achiving speeds up for algorithms and various optimizations that can be exploited in the Distributed and Hadoop Environment
\end{itemize}
\textbf{Technische Universit\"{a}t, Braunschweig} \hfill{\sl \small Summer 2013}\\
Online Triangulation and Navigation Using a Swarm of Simple Robots \hfill{\sl \small Guide: Prof. Dr. S\'{a}ndor P. Fekete}\\
\vspace{-17pt}
\begin{itemize}[itemsep = -0.75 mm, leftmargin=*]
    \item Researched on Maximum Area Triangulation with K Robots \& Minimum Robot Area Triangulation 
    \item Contributed to practical aspects of real life problmes involving minimizing oerall errror in navigation and locatlization, given minimum sensing capabilities such as Infrared Sensors
    \item Minimized the erorr by 80\% making it possible for the robots to locate and align with their neighbours
\end{itemize}
\vspace{-17pt}
\section*{Major Projects\xfilll[0pt]{0.5pt}}
\vspace{-7pt}
\textbf{Matsya, Autonomous Underwater Vehicle} \hfill{\sl \small 2012 - Present}\\
International Robosub, AUVSI \& US Naval Research \hfill{\sl \small Guide: Prof. Leena Vachanni}\\
AUV-IITB is an all student team working on the design and development of an unmanned autonomous underwater vehicle, “Matsya” and competes annually at the International AUVSI Robosub competition in San Diego, California
\vspace{-5pt}
\begin{itemize}[itemsep = -0.75 mm, leftmargin=*]
    \item \textbf{Team Leader}, 2014
    \vspace{-5pt}
        \begin{itemize}[itemsep = -0.75 mm, leftmargin=*]
            \item Leading a 30 member team to compete at world's largest Underwater Robotics Competition
            \item Managing operations, logistics and recruiting in a 4-tier Cross functional team 
            \item Designing a testing framework for hardware and software compatibilty across versions 
        \end{itemize}
    \item \textbf{Software Lead}, 2013
    \vspace{-5pt}
        \begin{itemize}[itemsep = -0.75 mm, leftmargin=*]
            \item Designed a modular architecture to integrate Doppler Velocity log(DVL) for navigation of the system
            \item Implemented Navigation using Markov Localization with Visual \& acoustic feedback
        \end{itemize}
    \vspace{-5pt}
    Semi-Finalist at Robosub 2012, 2013 \& 2014
\end{itemize}
\textbf{Undergraduate Dissertation} \hfill{\sl \small Ongoing}\\
Distributed Machine Learning \& Optimization\hfill{\sl \small Guide: Prof. Ganesh Ramakrishnan}\\
\vspace{-17pt}
\begin{itemize}[itemsep = -0.75 mm, leftmargin=*]
	\item Collectively using Struct SVM's, Latent SVM's and Multivariate Analysis to solve a problem on relationship prediction between entities. Using a Non-Decomposible Loss function to minimize training loss    
    \item Expoliting parallelism to improve speed and efficiency for the iterative process 
\end{itemize}
\textbf{Undergraduate Research Project} \hfill{\sl \small Ongoing}\\
Shape Analysis on Sphere Spaces\hfill{\sl \small Guide: Prof. Suyash P Awate}\\
\vspace{-17pt}
\begin{itemize}[itemsep = -0.75 mm, leftmargin=*]
	\item Used Principal Component Analysis and Prinicpal Geodesic Analysis on Hand and Heart Data Sets, to analyse various variations
    \item Working on Prinicipal Nested spheres for shape analysis
\end{itemize}
\vspace{-17pt}
\section*{Academic Projects \xfilll[0pt]{0.5pt}}
\vspace{-7pt}
\textbf{Texture Analysis} $|$ Prof. Suyash Awate \hfill{\sl \small Ongoing}\\
\vspace{-17pt}
\begin{itemize}[itemsep = -0.75 mm, leftmargin=*]
	\item Implementing texton extraction algorithm to identify representative patterns, to build a classifier to identify different settings of an image.
\end{itemize}
\textbf{Virtual Memory Implementation} $|$ Prof. D. M. Dhamdhere \hfill{\sl \small Spring 2014}\\
\vspace{-17pt}
\begin{itemize}[itemsep = -0.75 mm, leftmargin=*]
    \item Designed and developed a virtual memory system in Pranali OS handling page swap-in/swap-out during I/O
    \item Implemented Swap Space, Multilevel Page Table, TLB and page fault handling
\end{itemize}
\textbf{TeamFlowy: Team Management Webapp} $|$ Prof. Umesh Bellur\hfill{\sl \small Autumn 2013}\\
\vspace{-17pt}
\begin{itemize}[itemsep = -0.75 mm, leftmargin=*]
    \item A django based webapp to help manage teams that supports calendar view of tasks, blog posts and reminders
    \item Implemented analytics to track team performance using HighCharts Js library
\end{itemize}
\textbf{Sequence Alignment on GPU's} $|$ Prof. Bernard Menezes \hfill{\sl \small Autumn 2013}\\
\vspace{-17pt}
\begin{itemize}[itemsep = -0.75 mm, leftmargin=*]
         \item Implemented a Sequence Alignemnt problem on GPU's with parallel version of Needleman-Wunsch algorithm
          \item Tried the Parallel Prefix and Diagonal based apporach to solve the problem. The approch achieved $O(n)$ complexity as compared to $O(n^{2}$) in the Serial Version. Used CUDA as a platorm with on Nvidia GeForce GTX 780M   
\end{itemize}
\textbf{Ping Pong on FPGA} $|$ Prof. Ashwin Gumaste\hfill{\sl \small Spring 2013}\\
\vspace{-17pt}
\begin{itemize}[itemsep = -0.75 mm, leftmargin=*]
    \item Designed and implemented a \emph{double-player} version of Ping Pong in VHDL 
     \item Interfaced it with a \textbf{HDMI} display and  used manual inputs from the user to provide paddle movements, on Xilinx SPARTAN 6 Board 
\end{itemize}
\textbf{N Body Simulation} $|$ Prof. Varsha Apte\hfill{\sl \small Autumn 2012}\\
\vspace{-17pt}
\begin{itemize}[itemsep = -0.75 mm, leftmargin=*]
          \item Designed an simulation which showed the interaction between different particles under the effect of intermolecular forces like gravity, electrostatic and nuclear
          \item The simulation involved using the famous \emph{Barnes-Hut Algorithm} to optimize on computation
\end{itemize}
\vspace{-18pt}
\section*{Other Projects\xfilll[0pt]{0.5pt}}
\vspace{-7pt}
\textbf{ITSP} \\
\textbf{Statistical Data Analysis}\\
\textbf{Compiler for Context Free Grammar}\\
\textbf{Ping Pong on FPGA's}
\vspace{-18pt}
\section*{Positions of Responsibility\xfilll[0pt]{0.5pt}}
\textbf{Mentor, Department Academic Mentorship Programme}\hfill{\sl \small 2014-15}\\
Assisting students who are part of the Academic Rehabilitation Program at IIT Bombay in re-establishing themselves and taking complete course curriculum\\
\textbf{Technical Mentor} \hfill{\sl \small 2013}\\
Mentored teams for technical summer projects under Student Technical Activity Body(STAB) \\
\textbf{Coordinator Electronics Club,IIT Bombay} \hfill{\sl \small 2012-13}\\
Monitoring and mentoring technical activities all over the Institute. The body conducts lectures and sessions on Technical topics\\
\vspace{-18pt}
\section*{Extracurricular Activities\xfilll[0pt]{0.5pt}}
\vspace{-7pt}
	\begin{itemize}[itemsep = -0.75 mm, leftmargin=*]
		\item Participated at Workshop of the IEEE International Underwater Technology Symposium 2013 
        \item Participated in  Yahoo HackU  2013, created a web-appilication which could provide car-pooling services
        \item Created appilication to  auto compelete and correct song metadata, at Windows 8 Appfest 2012. Runnerup at Institute Hackathon conducted by WnCC.          
        \item Completed the enduro3 night trek, 16kms long conducted by NEF, Pune
		\item Completed a yearlong course on Badminton offered by the National Sports Organization
	\end{itemize}
\end{document}		
		
